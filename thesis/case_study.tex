\chapter{Case study}

\section{Zeroconf}
\subsection{System}
\subsection{Data}
\subsection{Model}
\subsection{Properties}
\subsection{Evaluation}

\section{Bees colony}
We study the collective behavior of a bee colony. Each bee in a colony
possibly stings after observing a threat in the surrounding environment, and
warn other bees by releasing a special substance, pheromone. By sensing the
pheromone released in the environment, other bees in the colony may also
sting. However, since stinging leads to the termination of an individual bee,
it reduces the total defense capability as well. With parametric Discrete-time
Markov chain as the model, we study how the actions of a single bee change
with regarding to the colony size of and pheromone amount. This project
propose and discuss Bayesian methods to estimate parameters of bees population
models. It shows that the proposed methods is scalable and can deliver
estimations of model parameters. Assume that each bee in a colony decides its next action (to sting or not to
sting) based only on the current state of the environment, and the number of
bees who sting or not sting can be modeled as a Markov process. To reduce the
complexity of the model, we make another assumption that the states of the bees
colony are observed after uniform time duration, hence the model is of
discrete-time.There are 3 assumptions on the system:
\begin{enumerate}
    \item Each bee release an unit amount of pheromone immediately after stinging.
    \item A bee dies after stinging and releasing pheromone. In the other words, no
          bee can sting more than once.
    \item Stinging behaviour only depends on the concentration of pheromone in the
          environment.
\end{enumerate}
Under these assumption, a bee colony can be viewed as a set of agents (bees)
interact with each other in a closed environment with the appearance of a factor
\textit{pheromone}. Afterward, the agent has probability to commit an action, namely \textit{sting}.
The agent is eliminated from environment after stinging.
Assume that we have a colony of $n$ bees initially. As aforementioned, an individual bee
is terminated after it stings. Thus, at the end of experiment, the number of
bees is $n'\in\{0,1,\ldots,n\}$. We model the bee colony with a DTMC
$\mathcal{M}=(S,\mathbf{P}, S_{init}, AP,L)$, such that
\begin{itemize}
    \item $|S_{init}|=1$
    \item There exists $n+1$ tSCCs which encode the population at the end of the experiment.
\end{itemize}
Semantics of Markov population models for bees colony are developed by
\cite{hajnal2019data}.
\subsection{System}
\subsection{Data}
\subsection{Model}
\subsection{Properties}
\subsection{Evaluation}


\section{SIR model}
\subsection{System}
\subsection{Data}
\subsection{Model}
\subsection{Properties}
\subsection{Evaluation}

