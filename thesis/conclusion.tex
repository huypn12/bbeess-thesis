\chapter{Conclusion}
We presented frameworks to perform data-informed parameter synthesis. The frameworks are tested
against different case studies and show that they can deliver both a set of satisfying parameter
values and an estimated parameter value close to the original value used to synthesize test data.
Therefore, these frameworks are applicable when we need to estimate the unknown attributes and
proactively verify the system against the property of interest.\\
In the case studies, the comparisons between rational-function-based RF-SMC and simulation-based
SMC-ABC-SMC show that the computational cost of SMC-ABC-SMC increases slower than that of RF-SMC,
while the two methods are comparable in terms of accuracy for parameter estimation. Therefore, we
suggest that given an arbitrary parametric DTMC, simulation-based method SMC-ABC-SMC is preferable
over the rational-function-based RF-SMC.\\
There are possible extensions to the presented frameworks, including but not limited to:
\begin{itemize}
      \item \textit{Statistical Model Checking}: we can use Absolute-Error Massart Bounds (proposed
            by Molyneux \cite{molyneux2020abc}, but currently not supported by PRISM) on Statistical
            Model Checking to achieve a better bound on the number of simulations.
      \item \textit{Bayesian Model Checking}: Jha \cite{jha2009bayesian}, and Zuliani
            \cite{zuliani2013bayesian} presents a novel method that improves Statistical Model
            Checking by using Bayesian inference.
      \item \textit{Sampling algorithms}: different sampling algorithms can be used to estimate
            posterior distribution. For example, PyMC3 library \cite{salvatier2016pymc3} uses
            No-U-Turn Sampling (Hoffman \cite{hoffman2014no}) algorithm by default, as it exploits
            the gradient of the likelihood to better approximate the posterior distribution.
      \item \textit{Implementation improvement}: Currently, StormPy technically prohibits our
            implementation from being parallelized since StormPy's core classes are not serializable
            \footnote{\url{https://github.com/moves-rwth/stormpy/issues/36}}. Porting to the C++
            language has several benefits by achieving the higher performance of C++ and exploiting the
            data-parallelism of Sequential Monte Carlo algorithm.
\end{itemize}