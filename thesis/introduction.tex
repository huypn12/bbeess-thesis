\chapter{Introduction}
\section{Motivation}

In different areas of research and application, the objects are to investigate how the number of
individuals changes under a certain set of assumptions. For instance
\begin{itemize}
      \item Number of online nodes in a distributed system.
      \item Number of surviving individuals in an epidemic model.
\end{itemize}
Markov population models \cite{kingman1969markov} are finite state-space, stochastic models that is
widely used in modeling complex and dynamic systems. In a Markov population model, each state
represents the number of individuals. Formally, in a Markov population model whose state space is
$S=(s_1,\ldots,s_n)$, there is a map $f:S\rightarrow\{0,1,\ldots,N-1,N\}$ where $N\in\mathbf{N}^*$
is the maximum number of individuals in the system.\\
In a Markov population models, such as Discrete-time Markov Chain, initial and transition
probabilities are known a-priori. In order to formalize unknown attributes of a system, we introduce
\textit{parametric Markov population models}. In a parametric Markov population model, each
transition is a function of parameters. As unknown features of the system are represented by
parameters, the following research questions are raised
\begin{itemize}
      \item Given a set of data collected by observing the system, how can we know about its
      parameters?
      \item Which values of parameters warrant that a certain property holds on our model?
\end{itemize}
In this thesis, we work with a specific parametric Markov population model, that is
\textit{parametric Discrete-time Markov Chain}. In order to answer the aforementioned research
questions, we presents a data-driven approach for parameter synthesis of parametric Discrete-time
Markov Chain.Parameter synthesis is an emerging research direction on probabilistic model checking.
Parameter synthesis problem is to find a set of parameter values to satisfy a certain reachability
property \cite{katoen2016probabilistic}.

\section{Contribution}
Contributions of this thesis are
\begin{itemize}
      \item Investigate a data-driven approach on parameter synthesis of parametric Discrete-time
      Markov Chain.
      \item Evaluate the scalability of the approach in cases of closed-form solution available and
      simulation.
      \item Compare the performances of optimization methods used to approximate posterior
      distribution.
\end{itemize}

\section{Structure of the thesis}

\begin{itemize}
      \item \textbf{Chapter 1} introduces motivations for the research topic.
      \item \textbf{Chapter 2} presents the theoretical background on probabilistic model checking,
      include discrete stochastic models and their  corresponding temporal logics.
      \item \textbf{Chapter 3}
      \item \textbf{Chapter 4} reviews the state-of-the-art works of other researchers on the
      problem of parameter synthesis.
      \item \textbf{Chapter 5} describes the benchmark.
      \item \textbf{Chapter 6} conclusion and future work.
\end{itemize}

